\chapter{Implementation}
\label{sec:implementation}

This chapter describes the implementation part of the thesis work. After presenting in Chapter~\ref{sec:lsa} the theoretical basis behind \gls{LSA}, and in Chapter~\ref{chapter:cluster_labeling} cluster labeling, theoretical application and specific implementation decisions are discussed here. All software tools and libraries which were used are pointed out, and code snipplets are given. Then in Chapter~\ref{chapter:evaluation}, test results are shown, and evaluation of the implementation is made. \\

\section{Clustering, labeling}
For the ontology development, we used Protege Ontology Editor 4.1~\footnote{\url{http://protege.stanford.edu/}, accessed December, 2010} from Stanfornd University. The ontology is a light-weight domain-specific ontology developed in OWL for CoreMedia \gls{CMS} domain. \\

\section{Tag Cloud implementation}
\label{sec:implementation:tag_cloud}
The implemented open source library used for tag cloud generation is called Opencloud\footnote{\url{http://opencloud.mcavallo.org/}, accessed December, 2010}, and is provided by Marco Cavallo.\\

Tag clouds can be generated automatically by using the most frequent words, by removing stop words, etc. preprosessing. They can also be custom made, by using important concepts retrieved after \gls{IR} task has been performed - e.g. on search results, on main concepts as in our case, based on concepts retrieved using \gls{LSA}. \\

\section{Tools used}
\label{sec:implementation:tools_used}
Airhead Research\footnote{\url{http://code.google.com/p/airhead-research/}} project was used as a semantic spaces Package which provides a java-based implementation of \gls{LSA}.\\

Apache Lucene\footnote{\url{http://lucene.apache.org/java/3_0_2/}, accessed December, 2010} was used as an indexing and search library.

\section{Advantages and drawbacks}
\label{sec:lsa:adv_disadv}

why am i using lsa instead of lda for example?\\

\begin{enumerate}
\item PLSA - characteristics, advantages, disadvantages
\item LDA - characteristics, advantages, disadvantages
\end{enumerate}

