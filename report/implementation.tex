\chapter{Implementation}
\label{sec:implementation}

This chapter describes the implementation part of the thesis work. After presenting in Chapter~\ref{sec:lsa} the theoretical basis behind \gls{LSA}, and in Chapter~\ref{chapter:cluster_labeling} cluster labeling, theoretical application and specific implementation decisions are discussed here. All software tools and libraries which were used are pointed out, and code snipplets are given. Then in Chapter~\ref{chapter:evaluation}, test results are shown, and evaluation of the implementation is made. \\


\section{LSA and concept visualization as a tag cloud}
Tag clouds can be generated automatically by using the most frequent words, by removing stop words, etc. preprosessing. They can also be custom made, by using important concepts retrieved after \gls{IR} task has been performed - e.g. on search results, on main concepts as in our case, based on concepts retrieved using \gls{LSA}. \\


\section{Cluster labeling}
The algorithm for cluster labeling Weighted Centroid Covering is implemented in Java, and can be seen in listing~\ref{topic_identification}.

For the ontology development, we used Protege Ontology Editor 4.1\footnote{\url{http://protege.stanford.edu/}, accessed December, 2010} from Stanfornd University. The ontology is a light-weight domain ontology developed in OWL for CoreMedia \gls{CMS} domain. \\


\section{Tools and libraries used in this work}
\label{sec:implementation:tools_used}

\subsubsection{Repository}
This thesis work is available online hosted in a reporitory under GitHub. Prototype implementation\footnote{\url{https://github.com/angievelinska/Tag-Cloud-Summarizer}}, project report\footnote{\url{https://github.com/angievelinska/Tag-Cloud-Summarizer/raw/master/report/thesis.pdf}} and \LaTeX~template\footnote{\url{https://github.com/angievelinska/Tag-Cloud-Summarizer/tree/master/report}} can be downloaded freely. \\

\subsubsection{LSA and SVD}
In this work the popular SVD C library\footnote{\url{http://tedlab.mit.edu/~dr/SVDLIBC/}, accessed December, 2010}  is used, created by Doug Rohde at the Massachusetts Institute of Technology. The implementation developed as a part of this project is Java-based, therefore for matrix computations the open source SVDLIBJ\footnote{\url{http://bender.unibe.ch/svn/codemap/Archive/svdlibj/}, accessed December, 2010} library is used, which is a Java-based port of SVD C, made available by Adrian Kuhn and David Erni at the University of Bern. \\

\subsubsection{k-means clustering algorithm}
Cluto clustering library\footnote{\url{http://glaros.dtc.umn.edu/gkhome/cluto/cluto/overview}, accessed October, 2010} is used as an implementation of the k-means clustering algorithm.\\

\subsubsection{Information retrieval}
S-Space project created by  Jurgensd and Stevens~(\cite{S-Space}) was used for constructing a semantic space of the text collection used for evaluation. \\

\subsubsection{Tag cloud}
The tag cloud used in this work is based on Opencloud\footnote{\url{http://opencloud.mcavallo.org/}, accessed December, 2010} project, authored by Marco Cavallo.\\

\subsubsection{Search and retrieval of search results}
Lucene\footnote{\url{http://lucene.apache.org/java/3_0_2/}, accessed December, 2010} is an open source search engine library, distributed by the Apache Software Foundation. It is used for indexing, query parsing and retrieval of search results. \\

\subsubsection{Building and deployment}
Maven\footnote{\url{http://maven.apache.org/}} is a software management tool, provided by Apache Software Foundation, which is used for building and testing the implementation prototype, and for deploying the web application module.


