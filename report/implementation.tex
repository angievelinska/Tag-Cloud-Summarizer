\chapter{Implementation}
\label{sec:implementation}

This chapter describes the implementation part of the thesis work. After presenting in Chapter~\ref{sec:lsa} the theoretical basis behind \gls{LSA}, and in Chapter~\ref{chapter:cluster_labeling} cluster labeling, theoretical application and specific implementation decisions are discussed here. All software tools and libraries which were used are pointed out, and code snipplets are given. Then in Chapter~\ref{chapter:evaluation}, test results are shown, and evaluation of the implementation is made. \\


\section{Tag cloud summarizer}

The prototype implementation is a web application, which outputs a tag cloud based on users' queries. Initially, a preprocessing of the document set is done, then a semantic space is constructed by running \gls{LSA} and performing dimensionality reduction during \gls{SVD}~(refer to Chapter~\ref{sec:lsa} for the terminology). The initial two stages are performed offline. Once the document set is indexed by a term-document  matrix, queries can be made by users. The terms in the documents closest to the queries in the semantic space, are input to the tag cloud. Querying and tag cloud generation are performed interactively online. \\

The desicion to implement the prototype as a web application is due to two factors. Firstly, tag clouds widely spread in online systems, and thus mainly used online. And secondly, the prototype should be implemented in the online documentation system DocMachine\footnote{\url{https://documentation.coremedia.com/}} at CoreMedia AG, Hamburg, and should also be available online. \\

\subsubsection{Preprocessing}
The documents used for evaluation are a part of the online documentation at CoreMedia AG, Hamburg\footnote{\url{https://documentation.coremedia.com/}}. Documents are stored as XML files, and CoreMedia \gls{CMS} Unified API\footnote{\url{https://documentation.coremedia.com/servlet/content/241548?language=en&version=5.2&book=coremedia:///cap/content/241548}} is used to access the plain text, as shown in Listing~\ref{doc_preprocessing}. Before executing \gls{LSA} in order to construct a semantic space from a document collection, a preprocessing the text corpus is made. Stop words, puncutuation and numbers are removed. No stemming is done, as terms from the semantic space are later used to generate a tag cloud. 

\subsubsection{LSA}
The LSA implementation uses the opensource Java-based S-Space library (see section~\ref{sec:implementation:tools_used}). Partial code for \gls{LSA} is given in listing~\ref{code_lat_analysis}. The complete source code can be found in the online repository where this work in available online\footnote{\url{https://github.com/angievelinska/Tag-Cloud-Summarizer/tree/master/summarizer/src/main/java/edu/tuhh/summarizer/lsa}}).

\subsubsection{Querying the semantic space}
After the semantic space has been constructed, and dimensionality reduction has taken place, queries can be made to find the documents closest to a given query, or the terms respectively, by querying the documents or term space. As a reminder, the term space consists of the matrix product: $U * \Sigma$, and the document space of the product: $\Sigma * V^{t} $ (fig.~\ref{lsa:truncated_svd}). 
In listing~\ref{querying} the source code for querying the semantic space is given.

\subsubsection{Tag cloud generation}
A tag cloud is generated using a collection of terms (or tags) with their corresponding weights. The weights are just the normalized term frequencies after a dimensionality reduction took place during \gls{LSA}. In listing~\ref{tag_cloud} is given the source code used for generating the tag cloud. \\

The Tag cloud summarizer is a tool that should aid the users of an \gls{IR} systems obtain a quick overview of the main concepts contained in search results they receive. As it will be used by users, its evaluation should in the best case be made by them. In Chapter~\ref{chapter:evaluation} an simple evaluation based on user feedback is provided. 

\section{Cluster labeling}
The algorithm for cluster labeling Weighted Centroid Covering is implemented in Java. It receives as input predefined clusters, and nominates cluster labels for each cluster. A detailed overview of the algorithm can be found in section~\ref{clustering:WCC}. 

\section{Tools and libraries used in this work}
\label{sec:implementation:tools_used}

\subsubsection{Repository}
This thesis work is available online hosted in a reporitory under GitHub. Prototype implementation\footnote{\url{https://github.com/angievelinska/Tag-Cloud-Summarizer}}, project report\footnote{\url{https://github.com/angievelinska/Tag-Cloud-Summarizer/raw/master/report/thesis.pdf}} and \LaTeX~template\footnote{\url{https://github.com/angievelinska/Tag-Cloud-Summarizer/tree/master/report}} can be downloaded freely. \\

\subsubsection{LSA and SVD}
In this work the popular SVD C library\footnote{\url{http://tedlab.mit.edu/~dr/SVDLIBC/}, accessed December, 2010}  is used, created by Doug Rohde at the Massachusetts Institute of Technology. The implementation developed as a part of this project is Java-based, therefore for matrix computations the open source SVDLIBJ\footnote{\url{http://bender.unibe.ch/svn/codemap/Archive/svdlibj/}, accessed December, 2010} library is used, which is a Java-based port of SVD C, made available by Adrian Kuhn and David Erni at the University of Bern. \\

\subsubsection{k-means clustering algorithm}
Cluto clustering library\footnote{\url{http://glaros.dtc.umn.edu/gkhome/cluto/cluto/overview}, accessed October, 2010} is used as an implementation of the k-means clustering algorithm.\\

\subsubsection{Information retrieval}
S-Space project created by  Jurgensd and Stevens~(\cite{S-Space}) was used for constructing a semantic space of the text collection used for evaluation. \\

\subsubsection{Tag cloud}
The tag cloud used in this work is based on Opencloud\footnote{\url{http://opencloud.mcavallo.org/}, accessed December, 2010} project, authored by Marco Cavallo.\\

\subsubsection{Search and retrieval of search results}
Lucene\footnote{\url{http://lucene.apache.org/java/3_0_2/}, accessed December, 2010} is an open source search engine library, distributed by the Apache Software Foundation. It is used for indexing, query parsing and retrieval of search results. \\

\subsubsection{CoreMedia CMS domain ontology}
For the ontology development, Protege Ontology Editor 4.1\footnote{\url{http://protege.stanford.edu/}, accessed December, 2010} was used, developed at Stanfornd University. The ontology is a light-weight domain ontology developed in OWL for CoreMedia \gls{CMS} domain (Appendix~\ref{appendix:onto} and listing~\ref{cm_ontology}). \\


\subsubsection{Building and deployment}
Maven\footnote{\url{http://maven.apache.org/}} is a software management tool, provided by Apache Software Foundation, which is used for building and testing the implementation prototype, and for deploying the web application module.


