\chapter{Implementation and evaluation}
\label{sec:implementation}

\begin{summary}
This chapter reports the implemented solution for the given thesis problem, gives discusses its advantages and disadvantages.
\end{summary}

It is a challenge by itself to come up with a sensible evaluation set for an IR implementation. ... Define here precision, recall, measures , how to measure LSA performance with  diff. k, clusters, cluster labelling. \\


The document collection consists of guides and manuals about CoreMedia \gls{CMS} 5.2.\\

\section{LSA implementation}
\label{sec:implementation:lsa_impl}
\gls{LSA} was applied to a collection of 11818 words in 4000 documents, all of which describe CoreMedia \gls{CMS} 5.2. The algorithm took  63552 ms for preprocessing and indexing of the whole document collection. \\

For the implementation of LSA this work uses the open LSA library which is part of Semantic Spaces Project\cite{S-Space}. It is developed at the Natural Language Processing Group at the University of California at Berkley (UCLA)\footnote{\url{http://code.google.com/p/airhead-research/}}.\\

The real difficulty of LSA is to find out how many dimensions to remove - the problem of dimensionality.\\

TODO:test if inluding only terms that occur in more than one document improved \gls{LSA} performance with respect to generating precise tag clouds.\\

\section{Tag Cloud implementation}
\label{sec:implementation:tag_cloud}
The implemented open source library used for tag cloud generation is called Opencloud\footnote{\url{http://opencloud.mcavallo.org/}}, and is provided by Marco Cavallo.\\

\section{Tools used}
\label{sec:implementation:tools_used}
Airhead Research\footnote{\url{http://code.google.com/p/airhead-research/}} project was used as a semantic spaces Package which provides a java-based implementation of \gls{LSA}.\\

Apache Lucene\footnote{\url{http://lucene.apache.org/java/3_0_2/}} was used as an indexing and search library.

\section{Advantages and drawbacks}
\label{sec:lsa:adv_disadv}

why am i using lsa instead of lda for example?\\

\begin{enumerate}
\item PLSA - characteristics, advantages, disadvantages
\item LDA - characteristics, advantages, disadvantages
\end{enumerate}

\section{Latest development in the field of LSA}
\label{sec:lsa:latest}
LSAView is a tool for visual exploration of latent semantic modelling, developed at Sandia National Laboratories \cite{CrDuSh09}.\\

at the end- improvements of lsa with the basics explained.\\

\section{Improvements}
only terms occuring in more than one documents have been included \\
stop words \\
compound words list \\
Lanczos algorithm for computation of SVD of large sparse matrices \\
Multi-threading computation in the project \\


