\chapter{Implementation and evaluation of results}
\label{sec:implementation}

\begin{summary}
This chapter reports the implemented solution for the given thesis problem, gives discusses its advantages and disadvantages.
\end{summary}


The document collection consists of guides and manuals about CoreMedia \gls{CMS} 5.2.\\

\section{LSA implementation}
\label{sec:implementation:lsa_impl}
\gls{LSA} was applied to a collection of 11818 words in 4000 documents, all of which describe CoreMedia \gls{CMS} 5.2. The algorithm took  63552 ms for preprocessing and indexing of the whole document collection. \\

For the implementation of LSA this work uses the open LSA library which is part of Semantic Spaces Project\cite{S-Space}. It is developed at the Natural Language Processing Group at the University of California at Berkley (UCLA)\footnote{\url{http://code.google.com/p/airhead-research/}}.\\

The real difficulty of LSA is to find out how many dimensions to remove - the problem of dimensionality.\\

\section{Tag Cloud implementation}
\label{sec:implementation:tag_cloud}
The implemented open source library used for tag cloud generation is called Opencloud\footnote{\url{http://opencloud.mcavallo.org/}}, and is provided by Marco Cavallo.\\

\section{Tools used}
\label{sec:implementation:tools_used}
Airhead Research\footnote{\url{http://code.google.com/p/airhead-research/}} project was used as a semantic spaces Package which provides a java-based implementation of \gls{LSA}.\\

Apache Lucene\footnote{\url{http://lucene.apache.org/java/3_0_2/}} was used as an indexing and search library.
