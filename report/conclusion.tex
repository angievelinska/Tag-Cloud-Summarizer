\chapter{Conclusion and outlook}
\label{sec:conclusion}
Modern search applications process huge amount of information, and have to cope with the ambiguity of search queries that the users enter in search engines\footnote{\url{http://www.hitwise.com/index.php/us/press-center/press-releases/2009/google-searches-apr-09/}, accessed December, 2010}. To address the problems of word ambiguity, \gls{LSA} was introduced in Chapter~\ref{sec:lsa} as an \gls{IR} technique. Clustering is another method used largely in \gls{IR} systems to group together similar documents, and thus present large amounts of search results in categories, which are better suited for browsing, than long search result lists. An important part of clutering used in search applications is how to label the clusters in a suitable and summarizing way, in order to increase clustering usability for humar users. Therefore, in Chapter~\ref{chapter:cluster_labeling} the algorithm for cluster labeling \gls{WCC} was reviewed. It nominates cluster labels from the terms in clusters which occur most frequently. However, clustering and \gls{WCC} examine neither the semantic structure in the text collection they process, nor the "meaning" of texts. As stated in the formal framework for cluster labels given in section~\ref{subsec:formal_framework}, labels need to be summarizing and hierarchically consistent. Therefore, an improvement of \gls{WCC} was proposed using external knowledge from an ontology during cluster label nomination.  And finally, in order to contribute to solving the issue of processing too many search results in search applications,  a web application was developed which provides users with a quick overview of the main concepts contained in returned results. The tool is called Tag cloud summarizer and uses \gls{LSA} to find the concepts closest to a given query. \\

\section{Future Work}
The goals set for this project and given in section~\ref{sec:goal_scope} were reached. The project can still be developed further. Below are outlined several proposals for future work. \\

\subsubsection{Tag cloud summarizer}
It remains for future work to improve the Tag cloud summarizer tool by implementing the tags in the cloud as hyperlinks, which return a document set where the correspoding term occurs most ferquently. Currently, the tag cloud displays only words, which are not hyperlinked to documents.  \\

The prototype remains to be tested using the whole online documentation available at CoreMedia AG. It remains for future work its implementation in DocMachine, the online documentation system at CoreMedia, where more extensive user feedback can be collected concerning its usability.

\subsubsection{Cluster labeling}
Other clustering methods can be investigated, such as \gls{STC}, which preserves the word order in documents, by presenting them in the form of a tree. Cluster labeling using \gls{STC} involves also using phrases as candidate labels, instead of separate terms, which improves usability, as compared 
to the investigates \gls{WCC} algorithm. \\

\subsubsection{Cluster labeling using external knowledge}
Due to time constraints, it remains for future work to evaluate and present experimental results on running \gls{WCC} algorithm for cluster labeling with external knowledge from an ontology. \\

\subsubsection{Testing}
A larger set of document can be used for evaluation and testing of our implementation. In this work tests are carried out on a document set consisting only of 15 documents in 3 categories. Research has shown~(\cite{dumais91improving}), however, that \gls{LSA} perform better when applied to document collections above 3000 documents, each of which larger than $\approx 60$ words, therefore testing the prototype on a larger document collection remains as future work. \\

