\chapter{Latent Semantic Analysis}
\label{sec:lsa}

\begin{summary}
This chapter presents Latent Semantic Analysis as a method for text processing, and defining the main concepts in texts.
\end{summary}

\section{1}
\label{sec:lsa:1}

\section{storage}
Text Processing and LSA\\
\\
Text processing:\\
-	retrieve documents from DB\\
-	tokenize texts\\
-	stem/lemmatize texts - this drops off as we will use the terms as a part of a tag cloud\\
-	stop wording\\
-	build SVD\\
-	post queries on the matrix\\
\\
The document collection consists of guides and manuals about CoreMedia Content Mangement System 5.2.\\
\\
Improving performance of LSA information retrieval method includes tf*idf weighting scheme, relevance feedback by implementing Tag Cloud, and choosing the number of dimensions for the reduced spacing. Stemming as a method for LSA improvement is not applied, as investigations showed at most modest improvements with this method.\\
\\
Library/implementation used for LSA is S-Spaces from Airhead Research project of UCLA (University of California at Los Angeles). The implemented algorithm for SVD is Lanczos, ported from SVDLIBC implementation by Doug Rohde from Tennessee University.\\
\\
Use the paper "Weight functions impact on LSA performance" by Preslav Nakov, Antonia Popova, Plamen Mateev - very nice concise description of LSA + analysis.\\
\\
IMPORTANT\\
I should test entropy and idf , as sometimes entropy global weighting function has a better performance.\\ 
\\
For text processing, Snowball project is used, from the laboratory of Martin Potter, the author of the infamous Porter Stemming algorithm.
!!! No stemming or lemmatization should be done on the input document collection, as the resulting terms/tags from LSA will be used in a TagCLoud!\\
\\

11818 words in word space\\
 63552ms to run LSA on 4000 documents\\
and IDEA blocks 9\\
\\
Due to the problem above, the process of SVD calculation has to be performed in a multi-threaded way, and the project has to be optimized with respect to performance, in order to be able to successfully run.

