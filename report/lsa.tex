\chapter{Latent Semantic Analysis}
\label{sec:lsa}

\begin{summary}
This chapter presents Latent Semantic Analysis as a method for text processing, and defining the main concepts in texts.
\end{summary}

\nomenclature{LSA}{Latent Semantic Analysis}Latent Semantic Analysis (LSA) uses a Singular Value Decomposition (SVD) to construct a..\\

There are three main factors that can influence the performance of LSA\cite{Nakov_weightfunctions}\cite{NakovBetterResultsLSI}:\\
\begin{itemize}
\item Frequency matrix transoformations (choice of weighting function)
\item Choice of dimensionality
\item Text preprocessing prior to SVD, choice of similarity measure (???)
\end{itemize}

Further, the choice of dimensionality is dependent upon the matrix transformations performed, as pointed out by Nakov in \cite{NakovBetterResultsLSI}.\\

\section{Latest development in the field of LSA}
LSAView is a tool for visual exploration of latent semantic modelling, developed at Sandia National Laboratories \cite{CrDuSh09}.\\

at the end- improvements of lsa with the basics explained.\\
why am i using lsa instead of lda for example?\\
\section{plan}
\label{sec:lsa:plan}
1. text processing and peculiarities; stemming, lemmatization, stop-wording\\
2. lsa and basics\\
3. weighting functions and their effect ot LSA results \cite{Nakov_weightfunctions}. \\
4. lsa used for information retrieval; lsa used for defining the main concepts in texts. precision vs. recall. \\
(first explain the basics of LSA, then explain how factors can influence lsa)\\
Several factors influence the quality of results which LSA delivers. These factors are pre-processing (removal of stop-words, stemming, lemmatization), frequency matrix transformations, choice of dimensionality, choice of similarity measure.\\
A study by Nakov, Popova, Mateev\cite{Nakov_weightfunctions} has summarized the influence of those factors on LSA, and has concluded that...\\

\section{storage}
Text Processing and LSA\\

Text processing:\\
-	retrieve documents from DB\\
-	tokenize texts\\
-	stem/lemmatize texts - this drops off as we will use the terms as a part of a tag cloud\\
-	stop wording\\
-	build SVD\\
-	post queries on the matrix\\
\\
The document collection consists of guides and manuals about CoreMedia Content Mangement System 5.2.\\
\\
Improving performance of LSA information retrieval method includes tf*idf weighting scheme, relevance feedback by implementing Tag Cloud, and choosing the number of dimensions for the reduced spacing. Stemming as a method for LSA improvement is not applied, as investigations showed at most modest improvements with this method.\\
\\
Library/implementation used for LSA is S-Spaces from Airhead Research project of UCLA (University of California at Los Angeles). The implemented algorithm for SVD is Lanczos, ported from SVDLIBC implementation by Doug Rohde from Tennessee University.\\
\\
Use the paper "Weight functions impact on LSA performance" by Preslav Nakov, Antonia Popova, Plamen Mateev - very nice concise description of LSA + analysis.\\
\\
IMPORTANT\\
I should test entropy and idf , as sometimes entropy global weighting function has a better performance.\\ 
\\
For text processing, Snowball project is used, from the laboratory of Martin Potter, the author of the infamous Porter Stemming algorithm.
!!! No stemming or lemmatization should be done on the input document collection, as the resulting terms/tags from LSA will be used in a TagCLoud!\\
\\

11818 words in word space\\
 63552ms to run LSA on 4000 documents\\
and IDEA blocks \\
\\
Due to the problem above, the process of SVD calculation has to be performed in a multi-threaded way, and the project has to be optimized with respect to performance, in order to be able to successfully run.\\
Keep only wht words common to at least 2 documents???\\

\section{Alternative approaches for LSA}
\begin{enumerate}
\item PLSA - characteristics, advantages, disadvantages
\item LDA - characteristics, advantages, disadvantages

